%% introduction.tex
%%

\chapter{Einleitung}
\label{ch:introduction}

Seit jeher werden Landkarten genutzt, um die topologische Beschaffenheit der Welt sowie, vor allem in Zeiten der Globalisierung, (über-)regionale Sachverhalte und Statistiken grafisch aufbereitet darzustellen.

Allerdings bestehen zwei Problematiken. Dreidimensionale Strukturen wie Globen, und damit auch deren Oberflächen, können aufgrund des "Verlustes\" einer Dimension grundsätzlich nur verzerrt auf zweidimensionalen Strukturen wie Papier oder Computer-Bildschirmen wiedergegeben werden. Außerdem gibt es unzählige Möglichkeiten, eine solche Verzerrung durchzuführen. Die Vereinheitlichung einer solchen Verzerrung in Form einer Funktion nennt man Projektion.

Die vorliegende Ausarbeitung beschäftigt sich mit solchen Projektionen. Nach einer theoretischen Einführung zu Projektionen wird gezeigt, wie Landkarten und Choreoplethenkarten in der JavaScript-Bibliothek D3.js angefertigt werden können.
%% introduction.tex
%%

\chapter{Einleitung}
\label{ch:introduction}

Seit jeher werden Landkarten genutzt, um die topologische Beschaffenheit der Welt sowie, vor allem in Zeiten der Globalisierung, (über-)regionale Sachverhalte und Statistiken grafisch aufbereitet darzustellen.

Allerdings besteht die Problematik, dass dreidimensionale Strukturen wie Globen (und damit auch deren Oberflächen) aufgrund des "Verlusts" einer Dimension grundsätzlich nur verzerrt auf zweidimensionalen Strukturen (wie Papier oder Computer-Bildschirmen) wiedergegeben werden können.

Die Gegebenheit, dass topologische Eigenschaften nur verzerrt wiedergegeben werden können und zusätzlich eine Vielzahl an Möglichkeiten für diese Wiedergabe existieren, bedingen die Problematik. Beispielsweise können mit einer Projektion topologische Strukturen flächentreu wiedergegeben werden, allerdings werden dann unter Umständen Streckenverläufe (wie zum Beispiel Reiserouten) verzerrt dargestellt. 

Die vorliegende Ausarbeitung beschäftigt sich mit dieser Problematik. Es werden zunächst Projektionen im Allgemeinen und danach deren Implementierung und Nutzung in der JavaScript-Bibliothek D3.js erläutert.

\section{Struktur}
\label{ch:introduction:sec:structure}

Für diese Arbeit lassen sich als Überschriften die Überschriften in verschiedenen Stufen verwenden.

\begin{verbatim}
\chapter{Einleitung}
\section{Struktur}
\subsection{}
\subsubsection{}
\end{verbatim}

Allerdings sollte man sich überlegen, ob man wirklich bis zur Stufe \verb|subsubsection| Überschriften benötigt.
